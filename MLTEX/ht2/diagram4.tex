%%%%%%%%%%%%%%%%%%%%%%%%%%%%%%%%%%%%%%%%%%%%%%%%%%%%%%%%%%%%%%%
%
% Welcome to Overleaf --- just edit your LaTeX on the left,
% and we'll compile it for you on the right. If you open the
% 'Share' menu, you can invite other users to edit at the same
% time. See www.overleaf.com/learn for more info. Enjoy!
%
% Note: you can export the pdf to see the result at full
% resolution.
%
%%%%%%%%%%%%%%%%%%%%%%%%%%%%%%%%%%%%%%%%%%%%%%%%%%%%%%%%%%%%%%%
% Kalman filter system model
% by Burkart Lingner
% An example using TikZ/PGF 2.00
%
% Features: Decorations, Fit, Layers, Matrices, Styles
% Tags: Block diagrams, Diagrams
% Technical area: Electrical engineering

\documentclass[a4paper,10pt]{article}

\usepackage[english]{babel}
\usepackage[T1]{fontenc}
\usepackage[ansinew]{inputenc}

\usepackage{lmodern}	% font definition
\usepackage{amsmath}	% math fonts
\usepackage{amsthm}
\usepackage{amsfonts}

\usepackage{tikz}

%%%<
\usepackage{verbatim}
\usepackage[active,tightpage]{preview}
\PreviewEnvironment{tikzpicture}
\setlength\PreviewBorder{5pt}%
%%%>

\begin{comment}
:Title: Kalman Filter System Model
:Slug: kalman-filter
:Author: Burkart Lingner

This is the system model of the (linear) Kalman filter. 

\end{comment}


\usetikzlibrary{decorations.pathmorphing} % noisy shapes
\usetikzlibrary{fit}					% fitting shapes to coordinates
\usetikzlibrary{backgrounds}	% drawing the background after the foreground

\begin{document}

\begin{figure}[htbp]
\centering
% The state vector is represented by a blue circle.
% "minimum size" makes sure all circles have the same size
% independently of their contents.
\tikzstyle{state}=[circle,
                                    thick,
                                    minimum size=1.2cm,
                                    draw=blue!80,
                                    fill=blue!20]

% The measurement vector is represented by an orange circle.
\tikzstyle{measurement}=[circle,
                                                thick,
                                                minimum size=1.2cm,
                                                draw=orange!80,
                                                fill=orange!25]

% The control input vector is represented by a purple circle.
\tikzstyle{input}=[circle,
                                    thick,
                                    minimum size=1.2cm,
                                    draw=purple!80,
                                    fill=purple!20]

% The input, state transition, and measurement matrices
% are represented by gray squares.
% They have a smaller minimal size for aesthetic reasons.
\tikzstyle{matrx}=[rectangle,
                                    thick,
                                    minimum size=1cm,
                                    draw=gray!80,
                                    fill=gray!20]

% The system and measurement noise are represented by yellow
% circles with a "noisy" uneven circumference.
% This requires the TikZ library "decorations.pathmorphing".
\tikzstyle{noise}=[circle,
                                    thick,
                                    minimum size=1.2cm,
                                    draw=yellow!85!black,
                                    fill=yellow!40,
                                    decorate,
                                    decoration={random steps,
                                                            segment length=2pt,
                                                            amplitude=2pt}]

% Everything is drawn on underlying gray rectangles with
% rounded corners.
\tikzstyle{background}=[rectangle,
                                                fill=gray!10,
                                                inner sep=0.2cm,
                                                rounded corners=5mm]

\begin{tikzpicture}[>=latex,text height=1.5ex,text depth=0.25ex]
    % "text height" and "text depth" are required to vertically
    % align the labels with and without indices.
  
  % The various elements are conveniently placed using a matrix:
  \matrix[row sep=0.5cm,column sep=0.5cm] {
    % First line: Control input
    &
        \node (x_1) [input]{$\mathbf{x}_{1}$}; &&
        
        \node (x_2)   [input]{$\mathbf{x}_2$};     &
        &
        
        \\
        
        \\
        % Second line: System noise & input matrix
        \node (w_1) [state] {}; &
        \node (w_2) [state] {};       &
        \node (w_k)   {$\cdots$}; &
       
        \node (w_k+1) [state] {}; &
        \node (w_n+1) [state] {};       &
        \\
        
        \\
        % Third line: State & state transition matrix
        \node (w_21) [state] {}; &
        \node (w_22) [state] {};       &
        \node (w_2k)   {$\cdots$}; &
        
        \node (w_2k+1) [state] {}; &
        \node (w_2n+1) [state] {};       &
                   \\
        
        \\
        % Fifth line: Measurement
        &&
        
        
        \node (z_1) [measurement] {$\mathbf{y}_{4}$}; &
        \\
    };
    
    % The diagram elements are now connected through arrows:
    \path[->]
        (x_1) edge (w_1)
        (x_1) edge (w_2)
        (x_1) edge (w_k+1)
        (x_1) edge (w_n+1)
        (x_2) edge (w_1)
        (x_2) edge (w_2)
        (x_2) edge (w_k+1)
        (x_2) edge (w_n+1)
        
        (w_1) edge (w_21)
        (w_1) edge (w_22)
        (w_1) edge (w_2k+1)
        (w_1) edge (w_2n+1)
        (w_2) edge (w_21)
        (w_2) edge (w_22)
        (w_2) edge (w_2k+1)
        (w_2) edge (w_2n+1)
        (w_k+1) edge (w_21)
        (w_k+1) edge (w_22)
        (w_k+1) edge (w_2k+1)
        (w_n+1) edge (w_2n+1)
        (w_n+1) edge (w_21)
        (w_n+1) edge (w_22)
        (w_n+1) edge (w_2k+1)
        (w_n+1) edge (w_2n+1)
        
        (w_21) edge (z_1)
        (w_22) edge (z_1)
        (w_2k+1) edge (z_1)
        (w_2n+1) edge (z_1)
        
        
        ;
    
    % Now that the diagram has been drawn, background rectangles
    % can be fitted to its elements. This requires the TikZ
    % libraries "fit" and "background".
    % Control input and measurement are labeled. These labels have
    % not been translated to English as "Measurement" instead of
    % "Messung" would not look good due to it being too long a word.
    \begin{pgfonlayer}{background}
   ,
                         \node [background,
                    fit=(x_1) (x_2),
                    label=left:Input:] {};
        \node [background,
                    fit=(w_1)  (w_2n+1),
                    label=left:Hidden layer] {};4
        \node [background,
                    fit=(z_1),
                    label=left:Output(no activation):] {};
    \end{pgfonlayer}
\end{tikzpicture}

\caption{Kalman filter system model}
\end{figure}

This is the system model of the (linear) Kalman filter. At each time
step the state vector $\mathbf{x}_k$ is propagated to the new state
estimation $\mathbf{x}_{k+1}$ by multiplication with the constant state
transition matrix $\mathbf{A}$. The state vector $\mathbf{x}_{k+1}$ is
additionally influenced by the control input vector $\mathbf{u}_{k+1}$
multiplied by the input matrix $\mathbf{B}$, and the system noise vector
$\mathbf{w}_{k+1}$. The system state cannot be measured directly. The
measurement vector $\mathbf{z}_k$ consists of the information contained
within the state vector $\mathbf{x}_k$ multiplied by the measurement
matrix $\mathbf{H}$, and the additional measurement noise $\mathbf{v}_k$.

\end{document}