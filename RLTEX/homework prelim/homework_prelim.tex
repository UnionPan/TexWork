\documentclass[11pt]{article}
\usepackage{amsmath,amssymb,amsthm}
\usepackage{mathrsfs}

\DeclareMathOperator*{\E}{\mathbb{E}}
\let\Pr\relax
\DeclareMathOperator*{\Pr}{\mathbb{P}}

\newcommand{\eps}{\varepsilon}
\newcommand{\inprod}[1]{\left\langle #1 \right\rangle}
\newcommand{\R}{\mathbb{R}}

\newcommand{\handout}[5]{
  \noindent
  \begin{center}
  \framebox{
    \vbox{
      \hbox to 5.78in { {\bf ECE-GY 9223 Reinforcement Learning} \hfill #2 }
      \vspace{4mm}
      \hbox to 5.78in { {\Large \hfill #5  \hfill} }
      \vspace{2mm}
      \hbox to 5.78in { {\em #3 \hfill #4} }
    }
  }
  \end{center}
  \vspace*{4mm}
}

\newcommand{\lecture}[4]{\handout{#1}{#2}{#3}{Scribe: #4}{ #1}}

\newtheorem{theorem}{Theorem}
\newtheorem{corollary}[theorem]{Corollary}
\newtheorem{lemma}[theorem]{Lemma}
\newtheorem{observation}[theorem]{Observation}
\newtheorem{proposition}[theorem]{Proposition}
\newtheorem{definition}[theorem]{Definition}
\newtheorem{claim}[theorem]{Claim}
\newtheorem{fact}[theorem]{Fact}
\newtheorem{assumption}[theorem]{Assumption}

% 1-inch margins, from fullpage.sty by H.Partl, Version 2, Dec. 15, 1988.
\topmargin 0pt
\advance \topmargin by -\headheight
\advance \topmargin by -\headsep
\textheight 8.9in
\oddsidemargin 0pt
\evensidemargin \oddsidemargin
\marginparwidth 0.5in
\textwidth 6.5in

\parindent 0in
\parskip 1.5ex

\begin{document}

\lecture{Homework Prelim}{Spring 2019}{Prof.\ Quanyan Zhu}{Yunian Pan}

\section{concepts of sequence and series}

\subsection{Problem 1.}

Proof:
\begin{enumerate}
\item {\em } -- $\lim \sup a_n = \lim_{N \to \infty} (\sup \{a_n: n> N\} )$, $\lim \inf a_n = \lim_{N \to \infty} (\inf \{a_n: n>N \})$.
\item {\em } -- suppose $\sup a_n = M_n$, the sufficiency: $M_n \leqslant b \implies \forall \epsilon > 0 \ \exists  N \in \mathbb{N} \ s.t. \forall n>N \ |a_n - M| < \epsilon  \implies a_n < M + \epsilon \implies a_n < b + \epsilon$; The necessity: $\forall \epsilon > 0 \ \exists  N \in \mathbb{N} \ s.t. \forall n>N \ a_n - b < \epsilon  \implies |a_n -b|< \epsilon \implies \lim_{n \to \infty} M_n \leqslant b $
\item {\em } -- Let $a_{n_k}$ denotes the subsequence of $a_n$ which converges to $\alpha$, it's clear that $n_k \geqslant k \forall k \in \mathbb{N}$, because $\exists N \in \mathbb{N} \ a_n < \alpha \ \forall n>N $, thus $\lim_{n_k \to \infty} a_{n_k} = \alpha$ since for all N we pick $\exists n_k > N$.
\end{enumerate}


\subsection{Problem 2.}

Proof: 
\begin{enumerate}
\item {\em } -- $\sum^{\infty}_{n = 1} a_n$ converges, $\lim_{n \to \infty}a_n = 0$, hence for somewhere $n > N$, we have $0<a_n < 1 \implies a_n^2 < a_n \implies \sum^{\infty}_{n = 1} a_n^2$ converges.
\item {\em } --  $\sum^{\infty}_{n = 1} |a_n|$ converges, $\lim_{n \to \infty}|a_n| = 0$, hence for somewhere $n > N$, we have $0<|a_n| < 1 \implies a_n^2 < |a_n| \implies \sum^{\infty}_{n = 1} a_n^2$ converges.
\item {\em } -- 1. $a_n = \frac{1}{n}$; 2. $a_n = \frac{2n+3}{n (n-1)}$
\end{enumerate}


\subsection{Problem 3.}


\section{Review concepts on the convergence of random variables}

\subsection{Blah blah blah}
Here is a subsection.

\subsubsection{Blah blah blah}
Here is a subsubsection. You can use these as well.

\subsection{Using Boldface}
Make sure to use \textbf{lots} of {\bf boldface}.

\paragraph{Question:}
How would you use boldface?

\paragraph{Example:}
This is an example showing how to use boldface to 
help organize your lectures.


\paragraph{Some Formatting.}
Here is some formatting that you can use in your notes:
\begin{itemize}
\item {\em Item One} -- This is the first item.
\item {\em Item Two} -- This is the second item.
\item \dots and here are other items.
\end{itemize}

If you need to number things, you can use this style:
\begin{enumerate}
\item {\em Item One} -- Again, this is the first item.
\item {\em Item Two} -- Again, this is the second item.
\item \dots and here are other items.
\end{enumerate}

\paragraph{Bibliography.}
Please give real bibliographical citations for the papers that we
mention in class. See below for how to include a bibliography section.
If you use BibTeX, integrate the .bbl file into your .tex
source. You should reference papers like this: ``Roger Myerson in \cite{1} demonstrated
in a bargaining application that ex post
efficiency can be incompatible with the
ex ante incentives of the parties individually.''
In general, the name of the authors should appear in text at most once 
(for the first citation); further citations look like: ``Our proof follows 
that of \cite{1}''.

Take a look at previous lectures (TeX files are available) to see the
details. A excellent source for bibliographical citations is
DBLP. Just Google DBLP and an author's name.


\bibliographystyle{alpha}

\begin{thebibliography}{42}

\bibitem{1}
Theodorou, Evangelos, Jonas Buchli, and Stefan Schaal. "A generalized path integral control approach to reinforcement learning." journal of machine learning research 11.Nov (2010): 3137-3181.	


\end{thebibliography}

\end{document}